\documentclass{beamer}

\usetheme{Goettingen}

%\usepackage{beamerthemesplit}
\usepackage{ucs}
\usepackage[utf8x]{inputenc}
\usepackage{ngerman}
\usepackage{graphicx}
\usepackage{tabularx}

\setbeamertemplate{footline}[frame number]

\begin{document}

\title[Freies Labor]{Ein etwas anderer \newline Hackerspace für Hildesheim}
\subtitle{"`Das Freie Labor"'}
\author[]{}
\date{13.11.2013}

%-----------------------------------------------------

\frame{\titlepage} 

%-----------------------------------------------------

\frame{\frametitle{Übersicht}\tableofcontents} 

%-----------------------------------------------------

\section{Allgemeines}

\subsection{Ziele}

\frame{\frametitle{Ziele} 
\begin{itemize}

\item Gründung eines technisch-kulturellen-künstlerischen Hackerspaces

\item "`Ein Hackerspace [...] ist ein Raum, in dem sich Hacker sowie an Wissenschaft, Technologie oder digitaler Kunst Interessierte treffen und austauschen können."'

\item Hildesheim: Kunst, Kultur, Technik

\end{itemize}}



\subsection{Was wollen wir sein?}

\frame{\frametitle{Was wollen wir sein? (1)} 
\begin{itemize}
\item Räumlichkeit
\item Verein
\item Werkstatt, freies Labor

\item    Offener Raum (räumlich (zweites Zimmer) und inhaltlich)

\item    Platz für Experimente

\item    erweitertes Wohnzimmer

\item    Clash of cultures?
\end{itemize}}

\frame{\frametitle{Was wollen wir sein? (2)} 
\begin{itemize}
\item    selbst gestalten

\item    Raumzeit

\item    Technik

\item    Kultur

\item    Politik

\item    Austausch

\item    Gemütlichkeit
\end{itemize}}

\subsection{Selbstverständnis}

\frame{\frametitle{Selbstverständnis} 
\begin{itemize}
\item Wir wollen ein freies Labor für Kunst, Kultur und Technik gründen! Gestalte mit uns eine Mischung aus Hackerspace und Atelier, die Raum für eigene Projekte und Austausch bieten soll. Wir wollen diese beiden Welten sowohl in einer Werkstatt als auch in einem gemeinsamen erweiterten Wohnzimmer dauerhaft in Hildesheim zusammen bringen. Bring Ideen mit oder lass' dich vom Know-how anderer Menschen inspirieren, um zusammen etwas aufzubauen.
\end{itemize}}



\section{Vorgehensweise}

\subsection{Grundsätzliches}

\frame{\frametitle{Grundsätzliches} 
\begin{itemize}
\item Grundsatz: Miteinbeziehung aller Stakeholder von Anfang an
\item Es gibt (im Idealfall) kein "`ihr"' und "`wir"'
\item Daher bisher keine tiefergehenden Planungen
\end{itemize}}


\subsection{Vergangenheit}

\frame{\frametitle{Was bisher geschah} 
\begin{itemize}
\item Erstes Planungstreffen 29.11.2013 bei Internet \& Tacos
\item Werbung für die heutige Veranstaltung
\item Heutige Vorbesprechung
\end{itemize}}

\subsection{Zukunft}

\frame{\frametitle{Weiteres Vorgehen} 
\begin{itemize}
\item Finden von interessierten Menschen, die bereit sind, einen monatlichen Beitrag zu zahlen
\end{itemize}

\begin{itemize}
\item Flyer-Werbung zur Vereinsgründung
\item Vereinsgründung 10.01.2014, 17 Uhr (unter Vorbehalt)
\end{itemize}

\begin{itemize}
\item Finden einer geeigeneten Immobilie
\end{itemize}

\begin{itemize}
\item \textbf{Wichtig: Es ist alles noch Veränderbar!}
\end{itemize}

}


\section{Diskussion}

\frame{\frametitle{Diskussion} 


}


\frame{\titlepage} 



\end{document}
